\documentclass[10pt,a4paper]{runcrypto}

\begin{document}

\part{Decentralised Identifiers (DID)}

\section{Overview}
A decentralised identifier (DID) is an identifier that isn't issued by a centralised body.  A DID is owned by an entity which can verify control of it using cryptography.

At the time of writing, there are solutions using blockchains/distributed ledgers that are being used to implement systems, providing self-sovereign identities.

Both private (\href{https://sovrin.org/}{Sovrin}) and public (\href{https://w3c-ccg.github.io/didm-btcr/}{BTCR}) blockchain solutions are in development and consolidate around the idea of Digital Identity Documents (DID) in which there has considerable progress in creating a standard (see https://w3c-ccg.github.io/did-spec/).

\section{Use cases}
\begin{itemize}
	\item Businesses selling restricted items such as alcohol or ciggarettes, may limit acceptable credentials to government issued ones
    \item Potential employers may limit educational credentials to recognised schools or universities.
    \item Door security systems may limit credentials to employees of companies renting space in shared office buildings.
    \item Homeowners may limit credentials to a tradesman, assigned to perform a particular job by a serviceing company.
\end{itemize}

\section{Specification}

There are 3 main components to Decentralised IDentifiers:
\begin{itemize}
	\item DID
	\item DID Document
	\item DID Method
\end{itemize}

They are used together to define an implementation.

\subsection{DID}
The DID is a unique reference that is used to root trust in the contents of the DID Document.  Control of a DID needs to be verified using public.

A DID is a unique identifier which must be persistent and immutable.

The DID method needs to ensure that two entities cannot claim ownership of the same DID.

There is a comparison in the specification of UUID's which are decentralised as a result of their uniqueness, in the same way that Bitcoin addresses are also generated to avoid collisions.

An entity can use a DID to present a DID Document to another entity and verify that they are the owner of that DID.  This mechanism is similar somewhat to providing a service with your email address and then clicking a verify link in an email that is subsequently sent to that address.

\subsection{DID Document}
A DID Document contains metadata about a DID.  The DID Document is what is exchanged during a interaction between two entities.

The DID Document contains a public key to verify ownership of the DID, using a DID Method.

The DID Document consists of the following parts: 
\begin{center}
	{\renewcommand{\arraystretch}{2}
		\begin{tabular}{ |p{2.5cm}|p{2.5cm}|p{6cm}| }
			\hline
			\multicolumn{2}{|l|}{context} &  \\
			\hline
			\multicolumn{2}{|l|}{id} & the DID \\
			\hline
			\multirow[t]{4}{*}{authentication} & id & Identifier for the authentication entry \\
			\cline{2-3}
			\cline{2-3} & type & type of the authentication entry \\
			\cline{2-3} & controller & the DID providing authentication \\
			\cline{2-3} & publicKeyPem & the public key of the DID \\
			\hline
		\end{tabular}
	}
\end{center}


\section{method}
Creating a DID requires that a DID can be linked to a public key.  The common implementation of this requires that a reference to the DID exists with a reference to the private key.

Bitcoin generates addresses that can be linked back to public keys and so having a UID generated in the same way that a bitcoin address is generated would satisfy the need to derive the UID from a public key.

\part{DID-LN}

\section{Introduction}
The DID-LN implementation aims to provide a method for storing the state of a DID using a lightning network channel id as the unique reference in the DID.  When the DID is to be revoked, the channel is closed.

\section{Problem Domain}
In this section, we explore the problems that need to be addressed in order to develop an identity system.  We look at other systems and how they have tackled the problem and also  provide some options around possible solutions for this implementation.

\subsection{Trust}
The bitcoin mantra is very apt here "Don't trust: verify". The goal of any decentralised identity system should be to minimise the amount of trust required and instead make any claims verifiable.

\subsubsection{Trust Triad}

DID's provide methods for verifying 3 key pieces of data in credentials:

\begin{itemize}
	\item The Owner is who they say they are
	\item The Provider has supplied the credential
    \item The credential is valid and has not been revoked
\end{itemize}

Each piece of data can be evaluated independently and adds to the information that the Requestor needs to make an informed decision about trust. It is possible for one or more of the endpoints needed for automatic verification to be unavailable and alternate methods for completing the trust triad to be used.

\subsubsection{Trust Models}
The trust model is very similar to legacy certificate authority systems where an agent will have a number of trusted root certificates. These are based on chains of trust which are typically pre-loaded on devices and not modified. An agent should be empowered to choose which providers are trusted.
When thinking about the nature of trust and the current implementation models, the following points come to mind:

\begin{itemize}
	\item Trust in the state of a system can be managed by software
	\item At some point the trust at an edge of the system has to resolve to a Trust Oracle of some kind.
	\item Secure internet sites, are trusted through a certificate issuing authority which is a form of Trust Oracle.
	\item The Sovrin network is an instance of Hyperledger Indy that acts as a Trust Oracle, where the user puts their trust in a decentralized consortium of Stewards.
	\item Trust oracles should be complementary to users making decisions on trust and not authoritative.
	\item Users should be empowered to make decisions on trust themselves.
\end{itemize}

\subsection{State Management}
One of the problems that creating a decentralised trust model faces is managing the state of the DID Document: is it valid or invalid at any point in time. Consider the following points on state:

\begin{itemize}
	\item For persistent or time limited DID Document (National Insurance Numbers/SSL certificates), the state can be rooted in the credential itself.
	\item Where a credential needs to be revocable, the state needs to be rooted outside of the credential.
	\item Hyperledger Indy addresses the revocation problem using Cryptographic Accumulators.
	\item The state of blockchains are saved in the UTXO set and not the chain itself which only provides a verifiable record of how the UTXO set came to its current state.
\end{itemize}

\section{Analysis}
The state management requires a link between two UTXO owners. This link is to be transient and given the scope of identity credential management will result in a high volume of links being created and destroyed.

\subsection{Blockchain selection}

Using the Bitcoin blockchain for this use case is not viable as it is not suited to handling the transaction volume needed. To address the tansaction volume requirement, a separate blockchain is required.

The elements project is an open source blockchain implementation that can be deployed as either a standalone blockchain or sidechain with 2 way peg and so will be the codebase to be deployed.

\subsubsection{Dedicated Blockchain}

If choosing to launch a new blockchain, the problem of issuing and trading tokens would present problems with price manipulation. Along with this, there may be backlash in the market with the launch of 'yet another blockchain' that needs justification during marketing cycles and with wider technical communities.

\subsubsection{Sidechain}

By having a sidechain, many of the issues with starting a dedicated blockchain are avoided. The sidechain builds on the reputation of the bitcoin network and tokens are tied to the price and liquidity of the BTC token. It is a lot easier to justify the existance of a sidechain and validity of the project that has no revenue generation scheme built in (i.e. creator premine).

\subsection{Storing State}
Transactional volume is one problem, the other is in the way that state is recorded. Current implementations use data in transactions to record state in the blockchain which, as discussed previously, does not fit with the idea of state being stored in the blockchain as immuntable and is a square peg in a round hole scenario. State needs to be stored in the UTXO set and so the following methods have been assessed.

\subsubsection{Multisig Wallet}

A link between two identities could be created with a multisig 1 of 2 wallet. Using BIP174 (Partially Signed Bitcoin Transaction) it should be possible to verify ownership of one of the keys without making an onchain transaction (also see https://blockstream.com/2019/02/04/standardizing-bitcoin-proof-of-reserves/).

\subsubsection{Lightning Channels}

An implementation of the multisig wallet solution is already in existence, with additional features added on, in the lightning network. It makes sense to use he existing functionality with lightning to build he solution on. There are channel identifiers that can be used to reference the link and a communication network that may be utilised to perform the state queries.

\subsection{Summary}

\subsubsection{Limitations}
For every channel, an on-chain transaction is required:
\begin{itemize}
	\item Transaction costs
	\item a number of confirmations are required before the channel is established.
\end{itemize}

Managing channels:
\begin{itemize}
	\item Channel factories can be used to open batches of channels.
	\item Closing a channel can be forced to close after 30 seconds.
\end{itemize}

\subsubsection{Trade-offs}

A couple of things can be done to limit the impact of the limitations: alternate public blockchains to Bitcoin could be used with faster block times and lower fees. Transaction costs and confirmations could be avoided altogetherby running a federated sidechain.

\subsubsection{In the wild}
In a broader sense, whenever friction is added to a transaction (requesting proof of age/salary) then there should be a cost to adding that friction. This will incentivise requests for proof to be used only when they are nessecary and for owners to adopt the system in the first place. Having the base layer on the lightning network would help facilitate micropayments being built into agents.

\section{Design}
What the Lightning Network method aims to provide is a solution to the state problem by using the lightning network to store the DID Document state.
\begin{itemize}
	\item Identity is rooted in ownership of UTXO's
    \item The owners of UTXO's can be verified using standard signing techniques.
    \item The lightning network extends the UTXO set and makes it possible to establish peer to peer links (channels) between UTXO's.
    \item Channels in the lightning network are used to show the state of a relationship between two UTXO's
\end{itemize}

When proving the credential, the owner is able to verify the ownership of the UTXO (public address) with their private key. The lightning network provides information on the existence of a channel with the provided Id and that the owners UTXO forms one end of the channel. The other end of the channel (the providers public address) will be verifyable via a Trust Oracle of the providers choosing (URL endpoint on a company endpoint).
\\
The DID Document can be revoked at any point by closing the channel between the provider and owner. This can be performed by either the provider OR the owner.

\subsection{document creation}
Entity, Issuer and Signer can all be the same entity, different entities or any mixture of.
\begin{enumerate}
	\item entity gets a DID using the DID Method
	\item entity requests a blank DID Document from the issuer
	\item the issuer creates a DID Document using a Document Descriptor Object
\end{enumerate}

\begin{sequencediagram}
	\newthread{entity}{:entity}
	\newinst[1.5]{issuer}{:issuer}
	\newinst[1.5]{method}{:method}
	\newinst[1.5]{signer}{:signer}
	
	\begin{call}{entity}{getDID(pubkey)}{method}{DID}
	\end{call}
	
	\begin{call}{entity}{createDoc()}{issuer}{doc}
	\end{call}
	
	\begin{call}{entity}{updateDoc(DID,auth,...)}{entity}{}
	\end{call}
	
	\begin{call}{entity}{sign(doc)}{signer}{doc}

		\begin{call}{signer}{validate(doc)}{method}{bool true/false}
		\end{call}

		\begin{call}{signer}{updateDoc(signature)}{signer}{}
		\end{call}
		
	\end{call}
	
\end{sequencediagram}

\end{document}

